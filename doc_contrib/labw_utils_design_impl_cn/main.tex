\documentclass[pdf,9pt]{beamer}

\newcommand{\lu}{\texttt{labw\_utils}}
\usepackage{xecjk}
\usepackage{noto}
\setCJKmainfont{NotoSerifCJKsc}[
    Extension = .otf ,
    UprightFont = *-Regular,
    BoldFont = *-Bold,
]
\setCJKsansfont{NotoSansCJKsc}[
    Extension = .otf ,
    UprightFont = *-Regular,
    BoldFont = *-Bold,
]
\setCJKmonofont{NotoSansMonoCJKsc}[
    Extension = .otf ,
    UprightFont = *-Regular,
    BoldFont = *-Bold,
]
% Math fonts
\usepackage{amsfonts}
\usepackage{amsmath}

% Source code highlighting
\usepackage{fancyvrb}
\VerbatimFootnotes
\usepackage{minted}
\fvset{
    numbersep=6pt,
    breaklines,
    breakanywhere,
    frame=single,
    numbers=left,
    tabsize=4,
    fontfamily=tt
}
\title{\lu:设计、实现与使用}
\subtitle{Design, Implementation and Usage of \lu}
\author{俞哲健}
\usepackage[bookmarks=true,bookmarksopen=true]{hyperref}
\setlength{\parskip}{1.2em}
\setcounter{tocdepth}{8}
\begin{document}
    \setbeamerfont{rmfont}{shape=\rmfamily}
    \usebeamerfont{rmfont}

    \begin{frame}
        \titlepage

        幻灯片基于 \LaTeX{} \& \href{https://latex-beamer.com/}{Beamer}
    \end{frame}

    \begin{frame}{目录}
        \tableofcontents[hideallsubsections]
    \end{frame}


    \section{为什么要开发 \lu{}?}

    \begin{frame}
        \sectionpage
    \end{frame}

    \begin{frame}{\lu{} 的开发目的}
        \lu{} 是一套为了基于\textbf{传统算法}的\textbf{生物信息学软件开发}而设计的 Python 支持库。其主要目的为:

        \begin{description}
            \item[代码复用] 在项目 A 里开发的代码可以不需要再在项目 B 里重新写一遍。
            \item[功能试验] 关于某个功能的新 idea 会在 \lu{} 内使用其已有的基础设施进行实验。
            \item[软件工程] 为生物信息学软件开发探索一般的模式。
        \end{description}
    \end{frame}

    \begin{frame}{\lu{} 的设计指标}
        \begin{description}
            \item[安全] 代码被广泛测试,没有故障。
            \item[轻量/便携] 代码需要的第三方软件极少,大部分模块只需要一个标准 Python 解释器即可运行。
            \item[高内聚,低耦合] 代码被按照不同功能层级分隔包装,相互之间互不干扰。
            \item[文档完全] 代码的包含详细的设计和实现文档、教程和索引以方便不同层次的开发者运行 \lu{}。
        \end{description}
    \end{frame}

    \section{\lu{} 的设计}

    \begin{frame}
        \sectionpage
    \end{frame}

    \subsection{运维和软工}

    \begin{frame}{\subsecname}
        \begin{enumerate}
            \item 使用 Python 3.8 开发,支持 Python 3.7 -- 3.13 各版本。
            \item 支持非官方 Python 实现(如 PyPy),支持侧载多种 JIT Python 加速库(如 Pyston)。
            \item 使用 PyTest 进行单元测试,能够自动生成测试报告和覆盖度报告。
            \item 使用 Sphinx 生成文档,可以基于 Python 代码内的注释生成网站。
            \item 使用 Tox 检测不同 Python 版本的兼容性。
            \item 使用 Setuptools、Build 和 Twine 自动化打包和发布流程。
        \end{enumerate}
    \end{frame}

    \subsection{主体部分}

    \begin{frame}{\subsecname}
        主体部分被分为以下几个子软件包:
        \begin{description}
            \item[\Verb|labw_utils| 自己] 声明版本号、前端、需要的依赖和一个类型提示兼容包。
            \item[\Verb|bioutils|] 用于生物信息学文件处理和基础算法实现的软件包。
            \item[\Verb|commonutils|] 通用 Python 软件开发支持包。实现了最常用的 Python 功能和标准库的增益。
            \item[\Verb|devutils|] 高级 Python 开发和软件工程支持包。
            \item[\Verb|mlutils|] 机器学习/深度学习支持包。
            \item[\Verb|stdlib|] 用于向下兼容的 Python 标准库。如,在 Python 3.7 使用 Python 3.9 版本引入的新函数。
        \end{description}
    \end{frame}

    \subsection{生物学文本文件处理}

    \begin{frame}{\subsecname}
        生物信息学软件开发过程中会使用大量诸如 FASTA、FASTQ 与 GTF 等生物学文本文件。

        \begin{itemize}
            \item 逐记录处理。此情况下,记录与记录之间的联系不存在或不重要,且计算机只需要记住当前记录即可。例如:
            \begin{enumerate}
                \item 过滤基因名为 \Verb|brca2| 的 GTF 记录。
                \item 过滤长度短于 120 的 FASTQ 记录。
            \end{enumerate}
            \item 处理记录集。此情况下,虽然记录与记录之间的顺序十分重要,且计算机需要记住整个数据集。例如:
            \begin{enumerate}
                \item 使用 FASTA 索引(FAI)查找 \Verb|chr5| 的位置。
                \item 按照空间顺序排序 GTF。
            \end{enumerate}
            \item 处理数据结构。此情况下,数据被按照层级关系组织成数据结构。例如:
            \begin{enumerate}
                \item 在 Gene-Isoform-Exon 三级结构中,统计所有 Exon 数小于 4 的 Isoform 数量(Exon 与 Isoform 都表示为单个记录,且不一定排序)。
            \end{enumerate}
        \end{itemize}
    \end{frame}

    \section{\lu{} 的使用}

    \begin{frame}
        \sectionpage
    \end{frame}

    \subsection{\lu{} 自己}

    \begin{frame}
        \subsectionpage
    \end{frame}

    \begin{frame}[fragile]{依赖声明}
        可以使用 \lu{} 提供的类注册依赖并在依赖未满足时报一个用户友好的错误。首先,在顶级 \Verb|__init__.py| 声明这个依赖:

        \begin{minted}{python}
from labw_utils import PackageSpecs, PackageSpec

PackageSpecs.add(PackageSpec(
    name="scipy",
    conda_name="scipy",
    pypi_name="scipy",
    conda_channel="conda-forge"
))
        \end{minted}

        之后便可在项目中依赖这个软件包的模块下写:

        \begin{minted}{python}
from labw_utils import UnmetDependenciesError

try:
    import scipy
except ImportError as e:
    raise UnmetDependenciesError("scipy") from e
        \end{minted}
    \end{frame}
    \begin{frame}[fragile]{类型提示}

        \Verb|labw_utils.typing_importer| 提供 Python 3.8 的类型提示并向上兼容。

        将代码中的

        \begin{enumerate}
            \item \mintinline{python}|from collections.abc import XXX|
            \item \mintinline{python}|from typing import XXX|
            \item \mintinline{python}|from typing_extension import XXX|
        \end{enumerate}

        改为:

        \mintinline{python}|from labw_utils.typing_importer import XXX|

    \end{frame}

    \subsection{\texttt{commonutils}:通用 Python 软件开发支持包}

    \begin{frame}
        \subsectionpage
    \end{frame}

    \subsubsection{\texttt{libfrontend}:基本命令行前端转发器}

    \begin{frame}[fragile]{\subsubsecname}

        设计目的:用于解决多个前端命令的解析;自动创建日志。

        假设模块 \Verb|yasim._main.generate_as_events| 和 \Verb|yasim._main.transcribe| 模块都会在命令行被调用。不使用 \lu{} 的命令行:

        \begin{minted}{shell-session}
$ python -m yasim._main.generate_as_events XXXXX
$ python -m yasim._main.transcribe XXXX
        \end{minted}

        使用 \lu{} 后的命令行:

        \begin{minted}{shell-session}
$ python -m yasim generate_as_events XXXXX
$ python -m yasim transcribe XXXX
        \end{minted}

    \end{frame}

    \begin{frame}[fragile]{\subsubsecname}

        \Verb|libfrontend| 能处理 \Verb|--help| 等参数并提供一个简单的帮助。

        \begin{minted}[fontsize={\fontsize{5}{6}\selectfont}]{shell-session}
$ python -m yasim help
2023-07-25 20:50:25,220 [INFO] yasim -- Yet Another SIMulator for Alternative Splicing and Realistic Gene Expression Profile ver. 3.1.5
2023-07-25 20:50:25,220 [INFO] Called by: /mnt/f/home/Desktop/yasim/src/yasim/__main__.py --help

This is frontend of `yasim` provided by `labw_utils.commonutils.libfrontend`.

SYNOPSYS: /mnt/f/home/Desktop/yasim/src/yasim/__main__.py [[SUBCOMMAND] [ARGS_OF SUBCOMMAND] ...] [-h|--help] [-v|--version]

If a valid [SUBCOMMAND] is present, will execute [SUBCOMMAND] with all other arguments

If no valid [SUBCOMMAND] is present, will fail to errors.

If no [SUBCOMMAND] is present, will consider options like:
    [-h|--help] show this help message and exit
    [-v|--version] show package version and other information

ENVIRONMENT VARIABLES:

    LOG_FILE_NAME: The sink of all loggers.
    LOG_LEVEL: Targeted frontend log level. May be DEBUG INFO WARN ERROR FATAL.

Use `lscmd` as subcommand with no options to see available subcommands.
        \end{minted}

    \end{frame}

    \begin{frame}[fragile]{\subsubsecname}

        \Verb|libfrontend| 提供 \Verb|lscmd| 命令,能动态列举所有的子命令并打印无法使用的子命令。

        \begin{minted}[fontsize={\fontsize{5}{6}\selectfont}]{shell-session}
$ python -m labw_utils.bioutils lscmd
2023-07-25 20:58:04,493 [INFO] ``labw_utils.bioutils`` -- Biological Utilities used in LabW projects ver. 1.0.3
2023-07-25 20:58:04,494 [INFO] Called by: /mnt/f/home/Desktop/yasim/deps/labw_utils/src/labw_utils/bioutils/__main__.py lscmd
2023-07-25 20:58:04,495 [INFO] Listing modules...
2023-07-25 20:58:05,455 [WARNING] Import official tqdm failed! will use builtin instead
2023-07-25 20:58:06,014 [WARNING] Subcommand describe_fasta_by_binning have unmet dependencies!
2023-07-25 20:58:06,334 [WARNING] You are importing from gene_view_v0_1_x -- This GTF parser is NOT safe and will be deprecated.
2023-07-25 20:58:07,062 [WARNING] Subcommand describe_gtf_by_binning have unmet dependencies!
2023-07-25 20:58:07,176 [WARNING] Subcommand get_exonic_depth have unmet dependencies!
describe_fastq -- Lite Python-implemented fastqc.
describe_gtf -- Get statistics about GTF files that can be parsed into a Gene-Transcript-Exon Three-Tier Structure
filter_gtf_by_attribute -- Filter GTF records by a specific attributes
generate_fake_fasta -- Create fake organism.
normalize_gtf -- Performs GTF normalization, etc.
sample_transcript -- Sample fraction of transcripts inside GTF file.
split_fasta -- Split input FASTA file into one-line FASTAs with one file per contig.
transcribe -- General-purposed stranded transcription, from reference genome to reference cDNA.
        \end{minted}
    \end{frame}
    
    \begin{frame}[fragile]{\subsubsecname}
    用法:

    \begin{enumerate}
        \item 将所有子命令放到当前包的子包下;如 \Verb|_main| 包。
        \item 将子命令的前端放到 \Verb|main| 函数内。
        \item 将命令行参数处理器(\Verb|argparse.ArgumentParser|)放到 \Verb|create_parser| 函数内;或写块状 docstring 并空出第一行。
        \item 在包级 \Verb|__main__.py|,写:
        
        \begin{minted}{python}
__version__ = ...  # 包的版本,允许导入
from labw_utils.commonutils.libfrontend import setup_frontend
description = "..."  # 包的“一句话描述”,允许导入

if __name__ == '__main__':
    setup_frontend(
        f"{__package__}._main", # 包的子命令所在位置的绝对包名
        description,
        __version__
    )
        \end{minted}
    \end{enumerate}
    \end{frame}

    \subsubsection{\texttt{str\_utils}:字符串反格式化器}

    \begin{frame}[fragile]{\subsubsecname}

        目前只实现了 \Verb|to_dict|,能够将非嵌套类 JSON 数据转为字典。示例:

        \begin{minted}[fontsize={\fontsize{5}{6}\selectfont}]{pycon}
>>> input_str = \
...     '\nCPU:\t2\nMEM:\t5.1\nPCIE:\t3rd Gen\n' \
...     'GRAPHICS:\t"UHD630\tRTX2070"\nUSB: "3.1"\nOthers:::info'
>>> to_dict(
...     input_str, field_sep=':', record_sep='\n', quotation_mark="\'\"", resolve_str=True
... )
{
    'CPU': 2, 'MEM': 5.1, 'PCIE': '3rd Gen',
    'GRAPHICS': 'UHD630\tRTX2070', 'USB': 3.1, 'Others': 'info'
}
        \end{minted}
    \end{frame}

    \subsubsection{\texttt{lwio}:类 Java 的 IO 封装}

    \begin{frame}{\subsubsecname}
        一坨大便;不要用它。
    \end{frame}

    \subsubsection{\texttt{serializer}:通用配置文件类型的(反)序列化器}

    \begin{frame}{\subsubsecname}
        尚未完成;不要用它。
    \end{frame}
    
    \subsubsection{\texttt{importer}:条件导入}

    \begin{frame}{\subsubsecname}
        目前仅实现了 \Verb|tqdm| 的条件导入。可以在 \Verb|tqdm| 未安装或标准错误流(Standard Error,stderr)不是远程打字机(Tele-TypeWriter,TTY)时换成更安静的版本避免污染日志。

        用法:将 

        \mintinline{python}|from tqdm import tqdm|

        替换为:

        \mintinline{python}|from labw_utils.commonutils.importer.tqdm_importer import tqdm|
    \end{frame}

    \subsubsection{\texttt{appender}:通用表格类数据文件追加器}

    \begin{frame}{\subsubsecname}
        使用动态加载技术,追加表格类数据;支持基于行的 buffering;支持多线/进程并行追加。

        支持的格式:

        \begin{itemize}
            \item TSV
            \item TSV.gz
            \item TSV.xz
            \item Apache Parquet(需要 FastParquet 和 Pandas)
            \item HDF5(需要 PyTable 和 Pandas)
            \item SQLite(需要 Pandas)
        \end{itemize}
    \end{frame}

    \begin{frame}[fragile]{\subsubsecname}

        使用示例:下列代码显示所有可用(依赖满足)的 Appender 类,加载其中一个类,实例化被加载的类并写入内容。
        \begin{minted}[fontsize={\fontsize{5}{6}\selectfont}]{pycon}
>>> from labw_utils.commonutils import appender
>>> list(appender.list_table_appender())
[('DumbTableAppender', 'Appender that does nothing.'),
 ('HDF5TableAppender', 'Append to HDF5 format. Requires PyTables and Pandas.'),
 ('TSVTableAppender', 'Append to TSV.'),
 ('LZ77TSVTableAppender', 'Append to GZipped TSV.'),
 ('LZMATSVTableAppender', 'Append to XZipped TSV.'),
 ('ParquetTableAppender', 'Append to Apache Parquet. Requires FastParquet.'),
 ('SQLite3TableAppender', 'Append to SQLite. Requires Pandas.')]
>>> TSVTableAppenderClass = appender.load_table_appender_class("TSVTableAppender")
>>> tsv_table_appender = TSVTableAppenderClass(
...     filename="1",
...     header=["h1", "h2", "h3"],
...     tac=appender.TableAppenderConfig()
... )
>>> tsv_table_appender.append([1, 2, 3])
>>> tsv_table_appender.flush()
>>> tsv_table_appender.close()
        \end{minted}

        将生成 \Verb|1.tsv|,其内容为

        \begin{Verbatim}
h1  h2  h3
1   2   3
        \end{Verbatim}
    \end{frame}

    \subsubsection{\texttt{stdlib\_helper.argparse\_helper}:文档增强的命令行参数处理器}

    \begin{frame}[fragile]{\subsubsecname}
        如下是典型的 Python 命令行处理器(Python 3.10 更改了一部分默认值):

        \begin{minted}[fontsize={\fontsize{5}{6}\selectfont}]{pycon}
>>> import argparse
>>> parser = argparse.ArgumentParser(prog="prog", description="description")
>>> _ = parser.add_argument("p", type=int, help="p-value")
>>> _ = parser.add_argument("-i", required=True, type=str, help="input filename")
>>> _ = parser.add_argument(
...    "-o", required=False, type=str,
...     help="output filename", default="/dev/stdout"
... )
>>> _ = parser.add_argument("--flag", action="store_true", help="flag")
>>> print(parser.format_help())
usage: prog [-h] -i I [-o O] [--flag] p

description

positional arguments:
  p           p-value

optional arguments:
  -h, --help  show this help message and exit
  -i I        input filename
  -o O        output filename
  --flag      flag

        \end{minted}
    \end{frame}
        
    \begin{frame}[fragile]{\subsubsecname}
        如下增强后的命令行参数处理器:

        \begin{minted}[fontsize={\fontsize{5}{6}\selectfont}]{pycon}
>>> from labw_utils.commonutils.stdlib_helper.argparse_helper import \
...     ArgumentParserWithEnhancedFormatHelp
>>> parser = ArgumentParserWithEnhancedFormatHelp(prog="prog", description="description")
>>> ...  # 参见上一张 Slide
>>> _ = parser.add_argument("--flag", action="store_true", help="flag")
>>> print(parser.format_help())
description

SYNOPSIS: prog [-h] -i I [-o O] [--flag] p

PARAMETERS:
  p
              [REQUIRED] Type: int; No defaults
              p-value

OPTIONS:
  -h, --help
              [OPTIONAL]
              show this help message and exit
  -i I
              [REQUIRED] Type: str; No defaults
              input filename
  -o O
              [OPTIONAL] Type: str; Default: /dev/stdout
              output filename
  --flag
              [OPTIONAL] Default: False
              flag

        \end{minted}

        可以看到,参数是否可选、类型、默认值等都被清晰地打印出来了。
    \end{frame}

    \begin{frame}[fragile]{\subsubsecname}
        额外地,此处理器增加了对 \Verb|enum.Enum| 的支持。使用如下额外的类函数 \Verb|from_name| 构建 Enum:

        \begin{minted}[fontsize={\fontsize{5}{6}\selectfont}]{python}
class SampleEnum(enum.Enum):
    """SampleEnum docstring"""
    A = 1
    B = 2

    @classmethod
    def from_name(cls, in_name: str) -> 'SampleEnum':
        for choices in cls:
            if choices.name == in_name:
                return choices
        raise TypeError
SampleEnum.A.__doc__ = "A doc"
SampleEnum.B.__doc__ = None
        \end{minted}

        (未完待续)
    \end{frame}
    
    \begin{frame}[fragile]{\subsubsecname}
        (续)

        之后使用:

        \begin{minted}[fontsize={\fontsize{5}{6}\selectfont}]{pycon}
>>> parser = ArgumentParserWithEnhancedFormatHelp(...)
>>> _ = parser.add_argument(
...     "-e",
...     required=False,
...     type=SampleEnum.from_name,
...     choices=SampleEnum,
...     help="Choose between A and B",
...     default=SampleEnum.A
... )
>>> print(parser.format_help())
...  # 前面的部分被省略了
  -e {A,B}
              [OPTIONAL] Type: SampleEnum; Default: A
              Choose between A and B
              CHOICES:
                  A -- A doc
                  B -- None
>>> parser.parse_args(["-e", "A"])
Namespace(e=<SampleEnum.A: 1>)
        \end{minted}

        可以看到,Enum 被显示成两个选项,且它们的 docstring 被包含在了帮助里;提供的参数被正确处理成了 Enum 内对应的值。
    \end{frame}

    \subsubsection{\texttt{stdlib\_helper.enum\_helper}:支持从标识符或值创建 Enum 内对应值的 Enum}
    \begin{frame}{\subsubsecname}
        参见上文。
    \end{frame}

    \subsubsection{\texttt{stdlib\_helper.itertools\_helper}:更丰富的 Iterable 支持}
    \begin{frame}[fragile]{\subsubsecname}
        类似 \Verb|str.translate| 的字典和列表翻译器(不认识的元素会被自动跳过):

        \begin{minted}[fontsize={\fontsize{5}{6}\selectfont}]{pycon}
>>> dict_translate({'A':1, 'B':2, 'C':3}, {'A':'a', 'B':'b'})
{'a': 1, 'b': 2, 'C': 3}
>>> list_translate(['A', 'B', 'C'], {'A':'a', 'B':'b'})
['a', 'b', 'C']
        \end{minted}

        更强大的滑动窗口支持:

        \begin{minted}[fontsize={\fontsize{5}{6}\selectfont}]{pycon}
>>> list(window([1, 2, 3], 2, "padd_front", 0))
[(1, 2), (0, 3)]
>>> list(window([1, 2, 3], 2, "padd_back", 0))
[(1, 2), (3, 0)]
>>> list(window([1, 2, 3], 2, "error", 0))
Traceback (most recent call last):
    ...
ValueError
>>> list(window([1, 2, 3], 2, "ignore", 0))
[(1, 2), (3,)]
>>> list(window([1, 2, 3], 2, "truncate", 0))
[(1, 2)]
        \end{minted}

        以及类似 R 的 \Verb|head| 与 \Verb|tail| 函数。
    \end{frame}
    
    \subsubsection{\texttt{stdlib\_helper.parallel\_helper}:带进度条的进程池}

    \begin{frame}{\subsubsecname}
        如无必要,使用 \Verb|joblib|。此模块内包含对 Jiblib 并行 map 的封装。
    \end{frame}
    
    \subsubsection{\texttt{stdlib\_helper.pickle\_helper}:带进度条和压缩支持的 Pickle}

    \begin{frame}[fragile]{\subsubsecname}
        \begin{minted}{pycon}
>>> import random
>>> random_arr = [random.random() for _ in range(1000)]
>>> pickle_fn = 'rd.pickle.xz'
>>> dump(random_arr, pickle_fn)
>>> unpickle_obj = load(pickle_fn)
>>> assert unpickle_obj == random_arr
        \end{minted}
    \end{frame}
    
    \subsubsection{\texttt{stdlib\_helper} 的其余部分}

    \begin{frame}[fragile]{\subsubsecname}
        \texttt{stdlib\_helper.shutil\_helper}:增强的 Shell Utils

        \hspace*{2em} 包含类似 GNU CoreUtils 的 \Verb|rm -rf|,\Verb|wc| 与 \Verb|touch| 实现。

        \texttt{stdlib\_helper.sys\_helper}:部分系统调用实现

        \hspace*{2em} 目前仅包含一个判断用户是否是管理员(Administrator 或 root)的函数。

        \texttt{stdlib\_helper.tty\_helper}:远程打字机协助。

        \hspace*{2em} 目前仅包含标准 ANSI 颜色。
    \end{frame}

    \subsection{\texttt{devutils}:开发者使用的高级内容}

    \begin{frame}
        \subsectionpage
    \end{frame}

    \subsubsection{\texttt{decorators}:高级装饰器实现}

    \begin{frame}[fragile,allowframebreaks]{\subsubsecname}
        主要包含以下几个函数:

        \begin{description}
            \item[\Verb|copy_doc|] 复制某个对象的 docstring 至目标对象。示例:
            \begin{minted}[fontsize={\fontsize{5}{6}\selectfont}]{python}
class Test:
    def foo(self) -> None:
        """Woa"""
        ...

    @copy_doc(foo)
    def this(self) -> None:
        pass
            \end{minted}

            此时有:

            \begin{minted}[fontsize={\fontsize{5}{6}\selectfont}]{pycon}
>>> Test.this.__doc__
'Woa'
            \end{minted}
        
            \item[\Verb|chronolog|] 一个可以向日志中插入函数调用时间、参数、返回时间和返回值的,作用于函数的装饰器。
            \item[\Verb|supress_inherited_doc|] 从被装饰的类或其实例中删除继承的成员的文档,当且仅当 \Verb|SPHINX_BUILD| 环境变量被设置时启动。注意:这将改变类的行为! 
            \item[\Verb|doc_del_attr|] 从类或其实例中删除被装饰的成员的文档,当且仅当 \Verb|SPHINX_BUILD| 环境变量被设置时启动。注意:这将改变类的行为! 
            \item[\Verb|create_class_init_doc_from_property|] 从类的属性中创建 \Verb|__init__| 函数文档。与属性同名的参数将被自动注释。示例:
            \begin{minted}[fontsize={\fontsize{5}{6}\selectfont}]{python}
@create_class_init_doc_from_property()
class TestInitDoc:
    _a: int
    _b: int
    def __init__(self, a: int, b: int):
        ...

    @property
    def a(self) -> int:
        """Some A value"""
        return self._a

    @property
    def b(self) -> int:
        """Some B value"""
        return self._b
            \end{minted}

            此时有:

            \begin{minted}[fontsize={\fontsize{5}{6}\selectfont}]{pycon}
>>> print(TestInitDoc.__init__.__doc__)
:param a: Some A value
:param b: Some B value

            \end{minted}
        \end{description}
    \end{frame}

    \subsubsection{\texttt{myst\_nb\_helper}:文档类帮助程序}

    \begin{frame}{\subsubsecname}
        尚未完成。
    \end{frame}

    \subsection{\texttt{mlutils}:机器学习/深度学习实用程序}

    \begin{frame}
        \subsectionpage
    \end{frame}

    \subsubsection{\texttt{ndarray\_helper}:PyTorch 与 Numpy 数组实用程序}

    \begin{frame}[fragile]{\subsubsecname}
        实现以下函数:

        \begin{description}
            \item[\Verb|scale_np_array|] 将 Numpy 数组线性缩放使其最大、最小值等于需求值。示例:
            
            \begin{minted}[fontsize={\fontsize{5}{6}\selectfont}]{pycon}
>>> scale_np_array(np.array([1,2,3,4,5]), out_range=(0, 1))
array([0.  , 0.25, 0.5 , 0.75, 1.  ])
            \end{minted}

            \item[\Verb|scale_torch_array|] 显然。
            \item[\Verb|describe|] 统计 Numpy 或 PyTorch 的基础统计量,支持离散值和连续值(若超过 10 重类别即视为连续)。示例:
            
            \begin{minted}[fontsize={\fontsize{5}{6}\selectfont}]{pycon}
>>> describe(np.array([0, 0, 1, 1]))
'ndarray[int64] with shape=(4,); uniques=[0 1]; mean=0.5'

>>> describe(np.array(np.random.uniform(0, 21, size=12000), dtype=int))
'ndarray[int64] with shape=(12000,); quantiles=['0.00', '5.00', '10.00', '15.00', '20.00']; mean=...'
            \end{minted}
        \end{description}
    \end{frame}

    \subsubsection{\texttt{mlutils} 的其余部分}

    \begin{frame}[fragile]{\subsubsecname}

        \texttt{io\_helper}:读写压缩的 Numpy 数据
        
        \hspace*{2em} 实现了 \Verb|read_np_xz| 与 \Verb|write_np_xz|。

        \texttt{torch\_layers}:实用 PyTorch 层

        \hspace*{2em} 实现了一个 \Verb|Describe| 层,这个层将传入矩阵传给下一层同时打印其 \Verb|describe| 信息。
    \end{frame}
    \subsection{\texttt{stdlib}:Python 标准库拷贝}

    \begin{frame}[fragile]{\subsecname}
        实现了以下功能:

        \begin{enumerate}
            \item Python 3.11 的 \Verb|tomllib|,是一个完整的 TOML 解析器。
            \item Python 3.10 的 \Verb|pkgutil.resolve_name|。但是这个函数的官方实现质量不佳。
        \end{enumerate}
    \end{frame}

    \subsection{\texttt{bioutils}:生物信息学相关实用程序}

    \section{目标与展望}

    \begin{frame}
        \sectionpage
    \end{frame}

    \begin{frame}{目标与展望}
        \begin{enumerate}
            \item 加强单元测试。
            \item 完成未完成的部分(推迟)。
            \item 将基本算法实现迁移到 \lu{}。
        \end{enumerate}
    \end{frame}
\end{document}
